\section{Conceptos}\label{conceptos}

El objetivo de la Probabilidad es modelar y trabajar con incertidumbre.
Dicha incertidumbre puede provenir de diversas fuentes, entre ellas:

\begin{itemize}
\tightlist
\item
  Estocasticidad del sistema modelado (e.g.~mecánica cuántica,
  escenarios hipotéticos con aleatoriedad, etc.).
\item
  Falta de observabilidad: los sistemas deterministas se muestran
  aparentemente estocásticos cuando no se pueden observar todas las
  variables que los afectan.
\item
  Modelización incompleta: el uso de un modelo que descarta parte de la
  información observada (un modelo simple pero incompleto puede ser más
  útil que uno absolutamente preciso).
\end{itemize}

La probabilidad se puede estudiar desde el enfoque frecuentista, donde
se trabaja con las tasas de ocurrencia de sucesos, o desde el enfoque
bayesiano, que trata de medir los grados de creencia, de certidumbre,
con los que ocurre un evento. En la práctica se puede partir del mismo
tipo de razonamientos y cálculos para trabajar con ambos tipos de
probabilidad.

En esta sección se realiza un recordatorio de conceptos necesarios para
trabajar con probabilidades en el resto del texto.

\defineb
Una \textbf{variable aleatoria} es una función medible
\(X:\Omega\rightarrow E\) donde \(\Omega\) es un espacio de probabilidad
y \(E\) un espacio métrico. \definee
\defineb
El par \((\Omega, \Sigma)\) donde \(\Omega\) es un conjunto y \(\Sigma\)
una \(\sigma\)-álgebra sobre \(\Omega\) es un \textbf{espacio medible}.
\definee
\defineb
Si \((\Omega, \mathcal{F})\) es un espacio medible y \(\mu\) es una
medida sobre \(\mathcal{F}\), entonces la terna
\((\Omega, \mathcal{F}, \mu)\) es un \textbf{espacio de medida}. Si
además se verifica \(\mu(\Omega)=1\), entonces se trata de un
\textbf{espacio de probabilidad}. \definee

Intuitivamente, una variable aleatoria representa una variable del
problema que puede tomar distintos valores, y la probabilidad con la que
se darán dichos valores puede ser descrita por una distribución de
probabilidad. Cuando notamos \(X:\Omega\rightarrow E\), interpretamos
que \(\Omega\) es el conjunto de todos los sucesos posibles, y los
estados que puede tomar la variable \(X\) vienen dados por su imagen,
\(X(\Omega)\subset E\). Se dice que \(X\) es \textbf{discreta} si
\(X(\Omega)\) es numerable (incluyendo el caso finito), y es
\textbf{continua} si \(X(\Omega)\) es no numerable.

Una distribución de probabilidad sobre una variable discreta \(X\) se
puede describir mediante una función de probabilidad (\emph{Probability
Mass Function}, PMF) \(P:X(\Omega)\rightarrow [0,1]\), que verifica
\(\sum_{x\in X(\Omega)} P(x)=1\).

Una distribución de probabilidad sobre una variable continua \(X\) se
puede describir mediante una función de densidad (\emph{Probability
Density Function}, PDF) \(p:X(\Omega)\rightarrow [0,1]\), que verifica
\(\int_{X(\Omega)} x dx=1\).

Cuando una distribución describe varias variables, puede interesar
conocer la distribución de un subconjunto de las mismas. Esta se
denomina \textbf{distribución marginal}, y se consigue sumando o
integrando a lo largo de todos los valores de las variables que no están
en el subconjunto. Por ejemplo, si \(X\) e \(Y\) son variables
discretas, se tiene \[P(x) = \sum_{y\in Y(\Omega)}P(x, y)~.\] Si son
continuas, entonces se verifica
\[P(x) = \int\limits_{Y(\Omega)}p(x, y)dy~.\]

\subsection{Probabilidad condicionada}\label{probabilidad-condicionada}

En ocasiones es útil representar la probabilidad de un suceso
condicionado a la ocurrencia de otro. Para ello se utilizan
\textbf{probabilidades condicionadas}, que se notan \(P(y|x)\) (donde
\(y\in Y(\Omega), x\in X(\Omega)\)) y se calculan mediante la siguiente
fórmula, suponiendo que \(P(x) > 0\):

\begin{equation}P(y|x)=\frac{P(y,x)}{P(x)}~.\label{eq:cond}\end{equation}

\subsubsection{Encadenando probabilidades
condicionadas}\label{encadenando-probabilidades-condicionadas}

Una distribución de probabilidad conjunta sobre varias variables se
puede descomponer como probabilidades condicionadas sobre una sola
variable:
\[P(x_1, \dots x_n) = P(x_1)\prod\limits_{i=2}^n P(x_i\mid x_1, \dots x_{i-1})~.\]

Esta expresión se deduce por inducción de la ecuación~\eqref{eq:cond}.

\subsection{Independencia e independencia
condicionada}\label{independencia-e-independencia-condicionada}

Dos variables aleatorias, \(X\) e \(Y\), son \textbf{independientes} si
la su probabilidad conjunta equivale al producto de sus probabilidades:
\[P(x,y)=P(x)P(y)\forall x\in X^{-1}(\Omega),y\in Y^{-1}(\Omega)~.\]

Además, se dice que son \textbf{condicionalmente independientes}
respecto a una variable \(Z\) si la distribución de probabilidad
condicionada se factoriza por \(X\) e \(Y\):
\[P(x,y|z)=P(x|z)P(y|z)\forall x\in X^{-1}(\Omega),y\in Y^{-1}(\Omega),z\in Z^{-1}(\Omega)~.\]

\subsection{Momentos: esperanza, varianza y
covarianza}\label{momentos-esperanza-varianza-y-covarianza}

La \textbf{esperanza} de una variable aleatoria \(X\) viene dada por las
expresiones siguientes, para variables discretas y continuas
respectivamente:
\[\mathrm E[X]=\sum_{x\in X^{-1}(\Omega)}xP(x);\quad \mathrm E[X]=\int_{X^{-1}(\Omega)}xp(x)dx~.\]

\textbf{Nota}: Todos los momentos se toman respecto de una variable
aleatoria y una distribución de probabilidad asociada, por lo que la
notación correcta sería \(\mathrm E_{X~P}[X]\). Sin embargo, se omitirá
excepto para prevenir ambigüedades.

Se puede definir la esperanza de una función \(f\) sobre los valores de
una variable aleatoria, del siguiente modo:
\[\mathrm E[X]=\sum_{x\in X^{-1}(\Omega)}f(x)P(x);\quad \mathrm E[X]=\int_{X^{-1}(\Omega)}f(x)p(x)dx~.\]

La \textbf{varianza} da idea acerca de cómo de diferentes entre sí son
los valores de una variables conforme se muestrean por su distribución
de probabilidad: \[\mathrm{Var}(X)=\mathrm E[(X-\mathrm E[X])^2]~.\]

La \textbf{covarianza} relaciona dos variables aleatorias, indicando la
medida en que están relacionadas linealmente y la escala de dichas
variables:
\[\mathrm{Cov}(X, Y)=\mathrm E[(X-\mathrm E[X])(Y-\mathrm E[Y])]~.\]

Para un vector de variables aleatorias, \(X=(X_1, \dots X_n)\), la
\textbf{matriz de covarianza} se define como una función matriz
\(n\times n\) dada por:
\[\mathrm{Cov}(X)_{i,j}=\mathrm{Cov}(X_i, X_j)~.\]

\section{Resultados de convergencia}\label{resultados-de-convergencia}

Sea \(d\) una distancia en \(\RR^k\) y sea
\(\{X_n:\Omega\rightarrow\RR^k\}\) una sucesión de variables aleatorias,
sea \(X:\Omega \rightarrow \RR^k\) una variable aleatoria.

\defineb
Se dice que \(X_n\) \emph{converge en probabilidad} a \(X\) si para cada
\(\varepsilon>0\) se tiene \(P(d(X_n, X)>\varepsilon)\rightarrow 0\). Lo
denotamos \(X_n\pconv X\). \definee

\defineb
Se dice que \(X_n\) \emph{converge casi seguramente} a \(X\) si se da la
convergencia puntual en un conjunto de medida 1:
\[X_n\asconv X\Leftrightarrow P\left(\lim_{n\rightarrow +\infty} d(X_n, X)=0\right)=1\]
\definee

Es un resultado conocido que \(X_n\pconv X\Rightarrow X_n\asconv X\).

\lemmab
\label{lm:convergencia-va} Si \(\{X_n\}\) es una sucesión de variables
aleatorias con varianza finita y se verifican las siguientes
condiciones:
\[\exists x\in \mathbb R:\lim_{m\rightarrow +\infty} \mathrm{E}[X_m]=x,\quad \lim_{m\rightarrow +\infty} \mathrm{Var}[X_m]=0,\]
entonces se tiene que \(X_m\pconv x\). \lemmae

\theob[Teorema de la aplicación continua]
\label{th:cont-map-conv} Sea \(\{X_n\}\) es una sucesión de variables
aleatorias y \(X\) una variable aleatoria, valuadas en un espacio
medible \(E\). Sea \(g:E\rightarrow F\) con \(F\) otro espacio medible.
Entonces, si \(g\) es continua casi por doquier, se tiene:

\begin{gather*}
  X_n\pconv X \Rightarrow g(X_n)\pconv g(X),\\
  X_n\asconv X \Rightarrow g(X_n)\asconv g(X).
\end{gather*}

\theoe

\section{\textasciitilde{}Herramientas de inferencia estadística
(?)\textasciitilde{}}\label{herramientas-de-inferencia-estaduxedstica}

\subsection{Estimadores
máximo-verosímiles}\label{estimadores-muxe1ximo-verosuxedmiles}
