El álgebra tensorial es una herramienta ampliamente utilizada en la
implementación de técnicas de aprendizaje automático, \textbf{POR QUÉ}
La fuente principal de esta sección es \textcite[capítulo 8]{treil2013}.

\section{Espacios duales}\label{espacios-duales}

\subsection{Funcionales lineales y el espacio
dual}\label{funcionales-lineales-y-el-espacio-dual}

\defineb
Un \textbf{funcional lineal} en un espacio vectorial finito dimensional
\(V\) sobre un cuerpo \(K\) es una aplicación lineal
\(L:V\rightarrow K\). \definee
\defineb
El \textbf{espacio dual} de un espacio vectorial finito dimensional
\(V\) es \(V^*=\{L:V\rightarrow K\text{ lineal}\}=\mathcal{L}(V, K)\).
\definee

\exampleb
Sea \(V=\mathbb R^n\) y consideramos
\((\mathbb R^n)^*=\left\{L:\mathbb R^n\rightarrow\mathbb R\text{ lineal}\right\}\).
Sabemos que toda aplicación lineal de \(\mathbb{R}^{n}\) en
\(\mathbb{R}^{m}\) se expresa, fijada una base, como una matrix
\(m\times n\), luego identificamos \((\mathbb{R}^{n})^{*}\) con matrices
\(1\times n\) (en la base usual). Evidentemente hay un isomorfismo entre
este conjunto y \(\mathbb{R}^{n}\):

\begin{align*}
  (\mathbb{R}^{n})^{*} \cong \mathcal{M}_{1\times n}(\mathbb{R})&\cong \mathbb{R}^{n} \\
  (m_{1} \dots m_{n}) &\mapsto (m_{1}, \dots m_{n})
\end{align*}

Este hecho se generaliza para cualquier cuerpo \(K\):
\((K^{n})^{*}\cong K^{n}\). \examplee

\subsubsection{Cambio de coordenadas}\label{cambio-de-coordenadas}

Sea \(V\) \(K\)-espacio vectorial, sean
\(A=\left\{a_{1}, \dots a_{n}\right\}, B=\left\{b_{1}, \dots b_{n}\right\}\)
bases de \(V\) donde \(n = \dim_{K}V\).

Introducimos la siguiente notación: dada una base \(B\) de \(V\), \(B'\)
de \(W\), \(L\in\mathcal L(V, W)\), notaremos \([L]_{B', B}\) a la
expresión matricial de \(L\) en las bases \(B, B'\). Si \(L\in V^{*}\)
notamos \([L]_{B}\).

Sabemos que la expresión de \(L\in V^{*}\) en la base \(B\) viene dada
por su imagen por los vectores de la base, y el cambio de coordenadas es
\([L]_B=[L]_A[I]_{A,B}\).

Recordamos también que el cambio de base de \(v\in V\) se realiza
mediante \([B]_B=[I]_{B,A}[V]_A\) y que \([I]_{B,A}=[I]_{A,B}^{-1}\).
Así, llamando \(S=[I]_{B,A}\) observamos que el cambio de base de los
vectores asociados a las filas \([L]_B, [L]_A\) es:

\[
  [L]_{B}^t=(S^{-1})^t[L]_{A}^t
\]

\begin{prop}
Dado $V$ espacio vectorial, si $S$ es la matriz de cambio de base de $A$ a $B$ entonces la matriz de cambio de base de $V^{*}$ es $(S^{-1})^t$.
\end{prop}

\begin{lemma}
  \label{lemma:vectorcero}
  Sea $v\in V$. Si $L(v)=0\forall L\in V^{*}$ entonces $v=0$. Como consecuencia, si $L(v_1)=L(v_2)\forall L\in V^{*}$ entonces $v_1=v_2$.
  
  \begin{proof}
    Sea $B$ base de $V$. Entonces $L(v)=[L]_B[v]_B$. Basta tomar $L_k=[0, \dots 0, \overset{(k)}{1}, 0, \dots 0]$ y comprobar que en ese caso $L_k[v]_B=0$ implica que la $k$-ésima coordenada de $[v]_B$ es 0. Repitiendo el mismo paso para cada $k$ tenemos que $v=0$.
  \end{proof}
\end{lemma}

\subsection{El segundo dual}\label{el-segundo-dual}

Puesto que \(V^{*}\) es un espacio vectorial, podemos considerar también
su dual, que notaremos \(V^{**}\). Comprobaremos que, de hecho,
\(V^{**}\) es isomorfo a \(V\) de una forma natural. Dado \(v\in V\)
podemos tomar \(L_{v}\in V^{**}\) dado por
\(L_{v}(f) = f(v)\forall f\in V^{*}\). Así, podemos construir una
aplicación del espacio \(V\) en su segundo dual,
\(T:V\rightarrow V^{**}\), dada de forma natural por
\(Tv=L_v\forall v\in V\).

Para ver que \(T\) es un isomorfismo, observamos que las dimensiones de
los espacios coinciden: \(\dim V^{**}=\dim V^{*}=\dim V\). Por tanto,
bastará con ver que \(T\) es inyectivo: veamos para ello que
\(\Ker T=\{0\}\). Dado \(v\in \Ker T\), tenemos que
\(\forall f\in V^{*} f(v)=L_v(f)=T(v)(f)=0\). Por el
\autoref{lemma:vectorcero}, se tiene que \(v=0\).

Nótese que el isomorfismo \(T\) no depende de la elección de una base en
\(V\).

\section{Funciones multilineales.
Tensores}\label{funciones-multilineales.-tensores}

\defineb

Sean \(V_1,\dots,V_p,V\) espacios vectoriales sobre un cuerpo \(\KK\).
Una \emph{aplicación multilineal} (\(p\)-lineal) con valores en V es una
función \(F:V_1\times \dots\times V_p\rightarrow V\), lineal en cada
variable. Es decir, para cada \(k\in \{1,\dots, p\}\) y fijado
\((v_1,\dots,v_{k-1},0,v_{k+1},\dots,v_p)\in V_1\times \dots\times V_p\),
se tiene que la aplicación que lleva
\(v_k\mapsto F(v_1,\dots,v_{k-1},v_k,v_{k+1},\dots,v_p)\) es lineal.

Notamos por \(\LL(V_1,\dots,V_p;V)\) a la familia de todas las
aplicaciones \(p\)-lineales de \(V_1\times \dots\times V_p\) en \(V\).

\definee

\defineb

Un \emph{tensor} o \emph{funcional multilineal} es una aplicación
multilineal con codominio \(\KK\),
\(F:V_1\times \dots\times V_p\rightarrow \KK\). La cantidad \(p\) se
denomina la \emph{valencia} del tensor.

\definee

En particular, un tensor de valencia 1 es un funcional lineal, y un
tensor de valencia 2 es una forma bilineal.

\exampleb

Sean \(V_1,\dots,V_p\) \(\KK\)-espacios vectoriales y sean
\(f_1\in V_1^{*},\dots f_p\in V_p^{*}\) funcionales lineales. Definimos
el funcional multilineal \(F:V_1\times\dots\times V_p\rightarrow\KK\)
dado por
\[F(v_1,\dots,v_p)=f_1(v_1)f_2(v_2)\dots f_p(v_p),\ v_i\in V_i,\ k=1,2,\dots,p.\]

El funcional \(F\) se denomina \emph{producto tensorial} de los
funcionales \(f_i\) y lo notamos
\(F=f_1\otimes f_2\otimes\dots\otimes f_p\).

\examplee

\remb

El conjunto de las aplicaciones multilineales es un \(\KK\)-espacio
vectorial, mediante las siguientes operaciones de suma y producto por
escalar: sean \(F_1, F_2\in \LL(V_1,\dots,V_p;V),\alpha\in\KK\)

\begin{align*}
(F_1+F_2)(v_1,\dots,v_p)&=F_1(v_1,\dots,v_p)+F_2(v_1,\dots,v_p),\\
(\alpha F_1)(v_1,\dots,v_p)&=\alpha F_1(v_1,\dots,v_p).
\end{align*}

\reme

\propb
\label{prop:base-funcionales} Sean \(V_1,\dots,V_p\) \(\KK\)-espacios
vectoriales con bases \(B^{(1)},\dots,B^{(p)}\) respectivamente. Notamos
\(b_i^{(k)}\) al \(i\)-ésimo elemento de la base \(B^{(k)}\).

Para cada \(k\in\{1,\dots,p\}\) y para cada \(i\in\{1,\dots,\dim V_k\}\)
sea \(f_{i}^{(k)}\) el funcional lineal de \(V_k^{*}\) definido por

\begin{align*}
    f_{i}^{(k)}(b_i^{(k)})&=1\\
    f_{i}^{(k)}(b_j^{(k)})&=0,\ j\neq i.
\end{align*}

La familia
\[B=\left\{f_{i_1}^{(1)}\otimes\dots\otimes f_{i_p}^{(p)},\ 1\leq i_k\leq\dim{V_k},\ k\in\{1,\dots,p\}\right\}\]
es una base del espacio \(\LL(V_1,\dots, V_p;\KK)\).

En particular,
\[\dim\LL(V_1,\dots, V_p;\KK)=(\dim V_1)\dots (\dim V_p).\]

\proofb

Dada \(F\in \LL(V_1,\dots, V_p;\KK)\) queremos expresarla de forma única
en función de los elementos de la familia \(B\), es decir, buscamos
coeficientes \(\alpha_{i_1,i_2,\dots,i_p}\in\KK\) tales que

\begin{equation}
  \label{eq:basetensor}
  F=\sum\limits_{i_k\in\{1,\dots,\dim V_k\}} \alpha_{i_1,i_2,\dots,i_p}f_{i_1}^{(1)}\otimes\dots\otimes f_{i_p}^{(p)}\ .
\end{equation}

Por la definición de los funcionales, se tiene que

\begin{equation}
  \label{eq:evalbasetensor}
  f_{i_1}^{(1)}\otimes\dots\otimes f_{i_p}^{(p)}\left(b_{j_1}^{(1)},\dots,b_{j_p}^{(p)}\right)=1\Leftrightarrow i_1=j_1,\dots,i_p=j_p
\end{equation}

y, en otro caso,
\[f_{i_1}^{(1)}\otimes\dots\otimes f_{i_p}^{(p)}\left(b_{j_1}^{(1)},\dots,b_{j_p}^{(p)}\right)=0\ .\]

Evaluando ahora \(F\) \eqref{eq:basetensor} en
\(b_{i_1}^{(1)},\dots,b_{i_p}^{(p)}\):

\begin{equation*}
F\left(b_{i_1}^{(1)},\dots,b_{i_p}^{(p)}\right)=\alpha_{i_1,\dots,i_p}
\end{equation*}

lo cual nos da la unicidad de los coeficientes, en caso de que existan.
La existencia se deduce definiendo

\begin{equation*}
\alpha_{i_1,\dots,i_p}:=F\left(b_{i_1}^{(1)},\dots,b_{i_p}^{(p)}\right),
\end{equation*}

de forma que la condición \eqref{eq:evalbasetensor} se mantiene para
todas las tuplas del tipo \(b_{j_1}^{(1)},\dots,b_{j_p}^{(p)}\). Así, se
tiene la descomposición que buscamos y \(B\) es una base.

\proofe

\prope

\section{Productos tensoriales}\label{productos-tensoriales}

\defineb
Sean \(V_1,V_2,\dots,V_p\) espacios vectoriales. El producto tensorial
de los espacios es el conjunto de funcionales multilineales
\(\LL(V_1^*,V_2^*,\dots,V_p^*;\KK)\), y lo notamos
\(V_1\otimes V_2\otimes\dots\otimes V_p\). \definee

\corb
Sean \(V_1,\dots,V_p\) \(\KK\)-espacios vectoriales con bases
\(B^{(1)},\dots,B^{(p)}\) respectivamente. Llamamos \(b_i^{(k)}\) al
\(i\)-ésimo elemento de la base \(B^{(k)}\) y observamos que podemos
definir el producto tensorial de elementos de \(V_1,\dots,V_p\)
viéndolos como funcionales de \(V_1^*,\dots,V_p^*\). Entonces, la
familia
\[B=\left\{b_{i_1}^{(1)}\otimes\dots\otimes b_{i_p}^{(p)},\ 1\leq i_k\leq\dim{V_k},\ k\in\{1,\dots,p\}\right\},\]
es una base del espacio \(V_1\otimes V_2\otimes\dots\otimes V_p\).
\proofb
Consecuencia de la \autoref{prop:base-funcionales} y el
isomorfismo \(V_k^{**}\cong V_k\). \proofe
\core

\remb
Dados \(v_1\in V_1,\dots v_p\in V_p\), para
\(v'_k\in V_k,\ k\in\{1,\dots,p\},\lambda,\mu\in\KK\) y cualesquiera
\(f_1\in V_1^*,\dots f_p\in V_p^*\) se tiene:

\begin{align*}
  (v_1\otimes v_2\otimes\dots\otimes(\lambda v_k &+ \mu v'_k)\otimes\dots\otimes v_p)(f_1,\dots,f_p)=\\
  f_1(v_1)\dots f_k(\lambda v_k &+ \mu v'_k) \dots f_p(v_p) =\\
  f_1(v_1)\dots (\lambda f_k(v_k) &+ \mu f_k(v'_k)) \dots f_p(v_p) = \\
  \lambda f_1(v_1)\dots f_k(v_k) \dots f_p(v_p) &+ \mu \lambda f_1(v_1)\dots f_k(v'_k) \dots f_p(v_p) =\\
  (\lambda v_1\otimes v_2\otimes\dots\otimes v_k\otimes\dots\otimes v_p &+\mu v_1\otimes v_2\otimes\dots\otimes v'_k\otimes\dots\otimes)(f_1,\dots,f_p)
\end{align*}

Hemos comprobado que la aplicación
\((v_1,v_2,\dots,v_p)\mapsto v_1\otimes v_2\otimes\dots\otimes v_p\) es
lineal en cada variable. \reme

\section{Tensores covariantes y
contravariantes}\label{tensores-covariantes-y-contravariantes}

Sean \(X_1,X_2,\dots X_p,V\) espacios vectoriales y sea \(V_k\) bien
\(X_k\) o bien \(X_k^*\), para cada \(k=1,2,\dots,p\).

\defineb
Decimos que una aplicación multilineal \(F\in \LL(V_1,V_2,\dots,V_p;V)\)
es \emph{covariante} en la \(k\)-ésima variable si \(V_k=X_k\) y
\emph{contravariante} en dicha variable si \(V_k=X_k^*\).

Si \(F\) es covariante (resp. contravariante) en todas las variables
decimos simplemente que es covariante (resp. contravariante). Si \(F\)
es covariante en \(r\) variables y contravariante en \(s\) variables,
decimos que es \(r\)-covariante \(s\)-contravariante, o de valencia
\((r, s)\). \definee

\exampleb
- Dado un espacio vectorial \(V\), un funcional \(f\in V^*\) es un
tensor 1-covariante. - Un vector \(v\in V\), visto en el doble dual
\(V^{**}\), es un tensor 1-contravariante. \examplee

\section{Los tensores en aprendizaje
automático}\label{los-tensores-en-aprendizaje-automuxe1tico}

\subsection{\texorpdfstring{Tensores como
\emph{multi-arrays}}{Tensores como multi-arrays}}\label{tensores-como-multi-arrays}
