\section{Introducción a R}\label{introducciuxf3n-a-r}

R es un lenguaje de programación dirigido al tratamiento de datos, y
como tal proporciona estructuras de datos y funcionalidades básicas para
representar y tratar problemas de minería de datos. Además, existe toda
una plataforma de paquetes para R denominada CRAN, que cuenta con
multitud de librerías que facilitan tareas muy diversas, desde lectura y
visualización de datos hasta el propio procesamiento mediante distintos
algoritmos.

Algunas de las técnicas más relevantes de Deep Learning no supervisado
están disponibles ya en paquetes como \texttt{h2o} \autocite{h2o},
\texttt{deepnet} \autocite{deepnet} o \texttt{darch} \autocite{darch}.
Sin embargo, hasta ahora ninguna herramienta para R ha incluido
mecanismos de visualización que faciliten entender el comportamiento de
las redes profundas no supervisadas o los modelos generados por ellas.

\section{\texorpdfstring{El paquete
\texttt{ruta}}{El paquete ruta}}\label{el-paquete-ruta}

Se ha desarrollado un paquete software, denominado \texttt{ruta}, con el
objetivo de proporcionar en una sola herramienta el mayor número de
estructuras de aprendizaje profundo no supervisado posible.

\subsection{Metodología de
desarrollo}\label{metodologuxeda-de-desarrollo}

Ágil \textbf{explicar}

\subsection{Componentes del paquete}\label{componentes-del-paquete}

\section{\texorpdfstring{Visualización:
\texttt{rutavis}}{Visualización: rutavis}}\label{visualizaciuxf3n-rutavis}

Una herramienta compañera a \texttt{ruta} es \texttt{rutavis}, una
aplicación con interfaz de usuario web que permite componer
visualizaciones de forma interactiva y compararlas.

\subsection*{Uso de la aplicación}\label{uso-de-la-aplicaciuxf3n}
\addcontentsline{toc}{subsection}{Uso de la aplicación}
