La teoría de la información estudia y cuantifica la información presente
en una señal. El origen de la misma es la búsqueda de códigos para
compresión de datos y la maximización de la tasa de transmisión en
comunicación. A causa de sus fundamentos y aplicaciones, es un campo de
estudio que se relaciona con muchos otros ámbitos, como la ingeniería
eléctrica, la estadística, la probabilidad, la física, la economía y la
informática.

A partir de lo estudiado en el capítulo anterior, si queremos añadir una
intuición acerca de la información que aporta un suceso, será razonable
que la cantidad de información sea menor cuanto más probable sea el
suceso dado. En el caso extremo, el suceso seguro no aporta información
alguna.

Realizaremos dos aplicaciones de esta teoría al problema que ocupa este trabajo. Por un lado, el concepto de entropía cruzada nos permitirá definir en la \autoref{sec:funciones-de-coste} las funciones de coste o funciones objetivo que se tratarán de optimizar mediante las técnicas de Deep Learning. Por otro, la desigualdad de la información nos aportará la intuición de que un modelo muy cercano a la distribución de los datos nos permitirá reducir mejor la dimensionalidad.

Para formalizar estas intuiciones, definiremos y haremos uso de
conceptos como la entropía. La fuente principal de este capítulo es
\textcite{coverit}.

\section{Entropía. Propiedades y
magnitudes}\label{entropuxeda.-propiedades-y-magnitudes}

\subsection{Concepto}\label{concepto}

\defineb
La \emph{entropía} \(H\) de una variable aleatoria discreta \(X\) con
distribución \(p(x)=P(X=x)\) viene dada por la siguiente expresión:
\[H(X)=-\sum_{x\in X^{-1}(\Omega)}p(x)\log p(x)~.\] $\log$ denota el logaritmo neperiano. Si se usa otra base
\(b\) para el logaritmo de la definición notaremos \(H_{b}\). \definee

Observamos que se puede expresar como la esperanza de una función de la
variable \(X\): \[H(X)=\E\left[\frac 1 {\log p(X)}\right]~.\]

Además, es interesante notar que \(H\) es un funcional de \(p\) en el
sentido de que no depende de los valores que tome la variable aleatoria,
sino únicamente de la probabilidad de los mismos. El significado que
aporta la entropía de una variable es la cantidad de información
esperada en un suceso. Por esto, las distribuciones cercanas a la
uniforme tienen mayor entropía que las que son casi determinísticas.

\subsection{Propiedades}\label{propiedades}

\lemmab
La entropía de una variable es siempre positiva o nula. \proofb
Efectivamente, puesto que \(0\leq p(x)\leq 1\) para todo \(x\), se tiene
que \(log(1/p(x))\geq 0\), y como consecuencia la esperanza de dicha
función de \(X\) es no negativa. \proofe
\lemmae

\lemmab
\(H_b(X)=\left(\log_b a\right)H_a(X)\) \label{lm:entropy-base-change}
\proofb
Es consecuencia del cambio de base de los logaritmos. \proofe
\lemmae

\subsection{Magnitudes}\label{magnitudes}

Según la base que se tome para los logaritmos, la escala de la entropía
varía, por lo que se está midiendo en una magnitud distinta.

\begin{itemize}
\tightlist
\item
  Si se toman logaritmos en base 2, entonces se habla de la entropía en
  \emph{bits}.
\item
  Si se toman en base \(e\), se está midiendo la entropía en
  \emph{nats}\footnote{Denominación de la unidad de medida de información análoga al \emph{bit} para base $e$.}.
\end{itemize}

Además, el cambio de base del
\autoref{lm:entropy-base-change}
nos permite convertir de una magnitud a otra: \[H=(\log 2) H_2~.\]

\section{Entropía conjunta y entropía
condicional}\label{entropuxeda-conjunta-y-entropuxeda-condicional}

\defineb
La \emph{entropía conjunta} \(H(X,Y)\) de dos variables aleatorias
discretas con distribución conjunta \(p(x,y)\) se define como

\begin{equation}H(X,Y)=-\sum_{x\in\Omega}\sum_{x\in\Omega'}p(x,y)\log p(x,y)~,\end{equation}
que también se puede expresar como
\begin{equation}H(X,Y)=-\E\left[\log p(x,y)\right]~.\end{equation}

\definee

\defineb
Si \((X,Y)\sim p(x,y)\), entonces se define la entropía condicional
\(H(Y\mid X)\) como

\begin{align}
  H(Y\mid X) &= \sum_{x\in\Omega}p(x)H(Y\mid X=x)\\
             &= -\sum_{x\in\Omega}p(x)\sum_{y\in\Omega'}p(y\mid x) \log p(y \mid x)\\
             &= -\sum_{x\in\Omega}\sum_{y\in\Omega'}p(x,y) \log p(y \mid x)\\
             &= -\E_{p(x,y)} \log p(Y \mid X)~.
\end{align}

\definee

Estas dos definiciones están relacionadas por el siguiente teorema.

\theob[Regla de la cadena de la entropía]
\[H(X,Y)=H(X)+H(Y\mid X)\]

\proofb

\begin{align*}
  H(X,Y)&=-\sum_{x\in\Omega}\sum_{x\in\Omega'}p(x,y)\log p(x,y)\\
        &=-\sum_{x\in\Omega}\sum_{x\in\Omega'}p(x,y)\log\left(p(x)p(y\mid x)\right)\\
        &=-\sum_{x\in\Omega}\sum_{x\in\Omega'}p(x,y)\log p(x) - \sum_{x\in\Omega}\sum_{x\in\Omega'}p(x,y)\log p(y\mid x)\\
        &=-\sum_{x\in\Omega}p(x)\log p(x) - \sum_{x\in\Omega}\sum_{x\in\Omega'}p(x,y)\log p(y\mid x)\\
        &=H(X)+H(Y\mid X)~.
\end{align*}

\proofe
\theoe

\corb
\[H(X,Y\mid Z)=H(X\mid Z)+H(Y\mid X,Z)\] \proofb
La prueba sigue pasos análogos al teorema anterior. \proofe
\core

\section{Entropía relativa, información mutua y entropía
cruzada}\label{entropuxeda-relativa-informaciuxf3n-mutua-y-entropuxeda-cruzada}

Si la entropía de una variable es una medida de la cantidad de
información requerida para explicarla, la entropía relativa de una
función de distribución a otra \(D(p\Vert q)\) mide la ineficiencia de
asumir que la distribución de una variable es \(q\) cuando la
distribución verdadera es \(p\), es decir, la longitud adicional de
código basado en \(q\) necesaria para describir la variable respecto de
un código basado en \(p\).

\defineb
La \emph{entropía relativa} o \emph{divergencia de Kullback-Leibler}
entre dos funciones de probabilidad \(p\) y \(q\) se define como
\[D(p\Mid q)=\sum_{x\in\Omega}p(x)\log\frac{p(x)}{q(x)}=\E_{p}\left[\log\frac{p(X)}{q(X)}\right]~.\]
\definee

En la definición anterior usamos la convención de que
\(0\log\frac 0 q=0\) y \(p\log\frac p 0=\infty\), basada en argumentos
de continuidad.

La entropía relativa adolece de algunas propiedades para ser considerada
una distancia entre distribuciones de probabilidad. En concreto, no es
simétrica y no satisface la desigualdad triangular.

\defineb
Sean \(X, Y\) dos variables aleatorias con función de probabilidad
conjunta \(p(x,y)\) y funciones de probabilidad marginal \(p(x)\) y
\(p(y)\) respectivamente. La \emph{información mutua} \(I(X;Y)\) es la
entropía relativa entre la distribución conjunta y el producto de las
distribuciones:

\begin{align*}
  I(X;Y)&=\sum_{x\in\Omega}\sum_{y\in\Omega'}p(x,y)\log\frac{p(x,y)}{p(x)p(y)}\\
        &=D(p(x,y)\Mid p(x)p(y))\\
        &=\E_{p(x,y)}\left[\log\frac{p(X,Y)}{p(X)p(Y)}\right]~.
\end{align*}

\definee

\theob
La información mutua es la reducción en la incertidumbre de \(X\) dado
el conocimiento de \(Y\). Por simetría además \(X\) da tanta información
sobre \(Y\) como \(Y\) sobre \(X\):
\[I(X;Y)=H(X)-H(X\mid Y)=H(Y)-H(Y\mid X)~.\] Además, como consecuencia:
\[I(X;Y)=H(X)+H(Y)-H(X,Y)~,\] en particular, \[I(X;X)=H(X)~.\] \proofb

\begin{align*}
  I(X;Y)&=\sum_{x\in\Omega}\sum_{y\in\Omega'}p(x,y)\log\frac{p(x,y)}{p(x)p(y)}\\
        &=\sum_{x\in\Omega}\sum_{y\in\Omega'}p(x,y)\log\frac{p(x\mid y)}{p(x)}\\
        &=-\sum_{x\in\Omega}\sum_{y\in\Omega'}p(x,y)\log p(x) + \sum_{x\in\Omega}\sum_{y\in\Omega'}p(x,y)\log p(x\mid y)\\
        &=-\sum_{x\in\Omega}p(x)\log p(x) -\left(- \sum_{x\in\Omega}\sum_{y\in\Omega'}p(x,y)\log p(x\mid y)\right)\\
        &=H(X)-H(X\mid Y)~.
\end{align*}

La consecuencia se deduce de la igualdad \(H(X,Y)=H(X)+H(Y\mid X)\).
\proofe

\theoe

Definimos una versión condicionada de la entropía relativa.

\defineb
La \emph{entropía relativa condicional}
\(D\left(p(y\mid x)\Mid q(y\mid x)\right)\) se define como

\begin{gather*}D\left(p(y\mid x)\Mid q(y\mid x)\right)=\\\sum_{x\in\Omega}p(x)\sum_{y\in\Omega'}p(y\mid x)\log\frac{p(y\mid x)}{q(y\mid x)}=\E_{p(x,y)}\left[\log\frac{p(Y\mid X)}{q(Y\mid X)}\right]
\end{gather*}

\definee

Un concepto similar al de entropía relativa es la entropía cruzada.
Simplemente, se considera la longitud completa de código necesaria para describir la variable en función de un código basado en la distribución $q$ frente a $p$, en lugar de considerar la longitud de código adicional.

\defineb
La \emph{entropía cruzada} entre dos funciones de probabilidad \(p\) y
\(q\) se define como
\[C(p, q)=H(X) + D(p\mid q)=\E_p\left[\log q(X)\right]~.\]
\definee

\note{En ocasiones se utiliza la notación \(H(p,q)\) para
hablar de entropía cruzada, pudiendo confundirse con la entropía
conjunta. En este texto se utilizará en su lugar \(C(p,q)\).}

\remb
Podemos tomar la entropía cruzada sobre una variable condicionada a otra
de forma natural. Consideremos \(X,Y\) variables aleatorias siguiendo
una distribución \(p\), sobre las que asumimos una distribución \(q\).
La entropía cruzada de \(p\) respecto de \(q\) para \(Y\) condicionada a
\(X\) será:

\begin{align*}
  C(p(y\mid x),q(y \mid x))=&H(Y\mid X) + D\left(p(y\mid x)\Mid q(y\mid x)\right)\\
                =&\sum_{x\in\Omega}p(x)\sum_{y\in\Omega'}p(y\mid x)\left(-\log p(y\mid x)
                  +\log\frac{p(y\mid x)}{q(y\mid x)}\right)\\
                =&-\sum_{x\in\Omega}p(x)\sum_{y\in\Omega'}p(y\mid x)\log q(y\mid x)\\
                =&-\E_p[\log q(Y\mid X)]~.
\end{align*}

\reme

\section{La desigualdad de la
información}\label{la-desigualdad-de-la-informaciuxf3n}

Vamos a estudiar una de las desigualdades esenciales en Teoría de la
Información, que nos dirá que la distribución que determina los códigos
más cortos para unos datos es la propia distribución de los datos. Para
ello, nos basaremos en una de las desigualdades más usadas en
matemáticas, la desigualdad de Jensen.

\theob[Desigualdad de Jensen]
Si \(f\) es una función convexa y \(X\) es una variable aleatoria,
entonces \[\E[f(X)]\geq f(\E[X])~.\]

\proofb
A continuación se expone una demostración para distribuciones discretas
con un número finito de puntos de masa, es decir, puntos donde la
probabilidad no es nula. Llamamos \(x_1, \dots, x_k\) a dichos puntos y
\(p_1, \dots, p_k\) a las respectivas probabilidades.

Procedemos por inducción en el número de puntos de masa. Para dos puntos
de masa, la desigualdad se expresa
\[p_1f(x_1)+p_2f(x_2)\geq f(p_1x_1+p_2x_2)~,\] que se deduce de la
definición de función convexa.

Supuesto el teorema cierto para distribuciones de \(k-1\) puntos de
masa, veámoslo para \(k\) puntos. Escribiendo
\(p'_{i}=\frac{p_{i}}{1-p_k}\) para \(i=1,\dots,k-1\), se tiene

\begin{align*}
  \sum_{i=1}^kp_if(x_i)&=p_kf(x_k)+(1-p_k)\sum_{i=1}^{k-1}p'_if(x_i)\\
                       &\geq p_kf(x_k)+(1-p_k)f\left(\sum_{i=1}^{k-1}p'_ix_i\right)\\
                       &\geq f\left(p_kx_k+(1-p_k) \sum_{i=1}^{k-1}p'_ix_i\right)\\
                       &=f\left(\sum_{i=1}^kP_ix_i\right)~,  
\end{align*}
donde la primera desigualdad se deduce de la hipótesis de inducción y la
segunda es consecuencia de la convexidad.

Por argumentos de continuidad se puede extender esta demostración a
distribuciones continuas. \proofe
\theoe

\theob[Desigualdad de la información]\label{th:information-ineq}
Sean \(p\), \(q\), funciones de probabilidad de una variable aleatoria
\(X\). Entonces, \[D(p\Mid q)\geq 0\] \proofb
Sea \(A=\{x\in\Omega :p(x)>0\}\) el soporte de \(p\). Entonces, notando
que la función \(-\log\) es convexa por concavidad del logaritmo,

\begin{align*}
  D(p\Mid q)&=\E_p\left[-\log\frac{q(X)}{p(X)}\right]\geq -\log\E_p\left[\frac{q(X)}{p(X)}\right]\\
            &=-\log\left(\sum_{x\in\Omega}p(x)\frac{q(x)}{p(x)}\right)=-\log\left(\sum_{x\in\Omega}q(x)\right)\\
            &=-\log 1= 0
\end{align*}

\proofe
\theoe

Las siguientes son algunas consecuencias directas de este resultado.

\corb[No negatividad de la información mutua]
Para cualesquiera dos variables aleatorias \(X, Y\) se tiene
\(I(X;Y)\geq 0\). \proofb
Por definición de información mutua. \proofe
\core

\corb
Para cualesquiera dos variables aleatorias \(X, Y\) y funciones de
probabilidad \(p\) y \(q\), \[D(p(y\mid x)\Mid q(y\mid x))\geq 0~.\]
\core

\corb
Para una variable aleatoria \(X\) y funciones de probabilidad \(p\)
(dada por la distribución de X) y \(q\), \[C(p, q)\geq H(X)~.\] De igual
forma, si \(Y\) es otra variable aleatoria con la misma distribución, se
tiene \[C(p(y\mid x), q(y\mid x))\geq H(Y\mid X)~.\] \core

Lo que nos dicen el \autoref{th:information-ineq} y sus consecuencias es que cada vez que asumimos un modelo sobre unos datos que no coincide con su verdadera distribución, necesitamos una mayor longitud de código que la óptima para representarlos. En nuestro caso, si vamos a tratar de reducir la dimensionalidad de un conjunto de datos, nos aporta la intuición de que un modelo muy ajustado a la distribución verdadera será más beneficioso a la hora de obtener una representación de menor dimensionalidad.