
Los objetivos que se plantearon al inicio de este trabajo fueron los siguientes:
\begin{enumerate}
\item
  Recopilación y estudio bibliográfico de los fundamentos matemáticos
  del Deep Learning, con especial enfoque en la teoría de la información
  y teoría de la probabilidad.
\item
  Desarrollo e implementación de un software que recopile las
  técnicas de Deep Learning orientadas a reducción de dimensionalidad
  más relevantes e incluya visualizaciones y facilidades para el
  análisis.
\item
  Análisis de las técnicas utilizadas en relación con los
  resultados visuales generados por la herramienta software.
\end{enumerate}

El primer objetivo se ha cumplido y se ha extendido notablemente. En este trabajo se estudian los conceptos matemáticos que fundamentan el Deep Learning en los capítulos \ref{ch:probability} y \ref{ch:information}, pero también se introducen conceptos que se aplican en las implementaciones concretas (\autoref{ch:tensors}). El \autoref{ch:probability} motiva además de forma teórica el problema de reducción de la dimensionalidad. Asimismo, el estudio teórico se extiende en los capítulos \ref{ch:learning} y \ref{ch:deep} hasta el análisis de los algoritmos utilizados en las técnicas de aprendizaje que se usan en la parte práctica del trabajo.

El segundo objetivo también se ha cumplido en un grado alto. Como se expone en el \autoref{ch:ruta}, se han desarrollado dos paquetes software responsables de implementar distintas técnicas de Deep Learning que permiten reducir la dimensionalidad y generar visualizaciones sobre los modelos, respectivamente. La herramienta no es totalmente exhaustiva, en el sentido de que no contiene todas las estructuras profundas descritas en este trabajo, pero sí que define además un marco de trabajo sencillo en el que es fácil añadir nuevas técnicas y usarlas.

El tercer objetivo consistía en conocer de qué forma los gráficos sobre los modelos generados nos podían dar información acerca del comportamiento de una red neuronal entrenada. En la \autoref{sec:rutavis} se muestran algunos ejemplos de estos gráficos e ideas acerca de lo que nos pueden explicar de las redes. Consideramos por tanto, que también se ha cumplido este objetivo en cierto grado.

A continuación se enumeran las materias del Doble Grado más relacionadas con este trabajo:
\begin{itemize}
\item Geometría I
\item Estadística Descriptiva e Introducción a la Probabilidad
\item Geometría II
\item Análisis Matemático II
\item Probabilidad
\item Fundamentos de Programación
\item Metodología de la Programación
\item Estructuras de datos
\item Algorítmica
\item Inteligencia Artificial
\item Aprendizaje Automático
\item Inteligencia de Negocio
\end{itemize}