\documentclass[,ignorenonframetext,compress]{beamer}
\setbeamertemplate{caption}[numbered]
\setbeamertemplate{caption label separator}{: }
\setbeamercolor{caption name}{fg=normal text.fg}
\beamertemplatenavigationsymbolsempty
\usepackage{lmodern}
\usepackage{amssymb,amsmath}
\usepackage{ifxetex,ifluatex}
\usepackage{fixltx2e} % provides \textsubscript
\ifnum 0\ifxetex 1\fi\ifluatex 1\fi=0 % if pdftex
  \usepackage[T1]{fontenc}
  \usepackage[utf8]{inputenc}
\else % if luatex or xelatex
  \ifxetex
    \usepackage{mathspec}
  \else
    \usepackage{fontspec}
  \fi
  \defaultfontfeatures{Ligatures=TeX,Scale=MatchLowercase}
    \setmainfont[]{Fira Sans}
\fi
\usetheme[]{metropolis}
\usecolortheme{metropolis}
\usefonttheme{serif} % use mainfont rather than sansfont for slide text
% use upquote if available, for straight quotes in verbatim environments
\IfFileExists{upquote.sty}{\usepackage{upquote}}{}
% use microtype if available
\IfFileExists{microtype.sty}{%
\usepackage{microtype}
\UseMicrotypeSet[protrusion]{basicmath} % disable protrusion for tt fonts
}{}
\ifnum 0\ifxetex 1\fi\ifluatex 1\fi=0 % if pdftex
  \usepackage[shorthands=off,main=]{babel}
\else
  \usepackage{polyglossia}
  \setmainlanguage[]{}
\fi
\newif\ifbibliography
\hypersetup{
            pdftitle={Reducción de la dimensionalidad en problemas de clasificación con Deep Learning},
            pdfauthor={Francisco David Charte Luque},
            pdfborder={0 0 0},
            breaklinks=true}
\urlstyle{same}  % don't use monospace font for urls
\usepackage{color}
\usepackage{fancyvrb}
\newcommand{\VerbBar}{|}
\newcommand{\VERB}{\Verb[commandchars=\\\{\}]}
\DefineVerbatimEnvironment{Highlighting}{Verbatim}{commandchars=\\\{\}}
% Add ',fontsize=\small' for more characters per line
\newenvironment{Shaded}{}{}
\newcommand{\KeywordTok}[1]{\textcolor[rgb]{0.00,0.44,0.13}{\textbf{#1}}}
\newcommand{\DataTypeTok}[1]{\textcolor[rgb]{0.56,0.13,0.00}{#1}}
\newcommand{\DecValTok}[1]{\textcolor[rgb]{0.25,0.63,0.44}{#1}}
\newcommand{\BaseNTok}[1]{\textcolor[rgb]{0.25,0.63,0.44}{#1}}
\newcommand{\FloatTok}[1]{\textcolor[rgb]{0.25,0.63,0.44}{#1}}
\newcommand{\ConstantTok}[1]{\textcolor[rgb]{0.53,0.00,0.00}{#1}}
\newcommand{\CharTok}[1]{\textcolor[rgb]{0.25,0.44,0.63}{#1}}
\newcommand{\SpecialCharTok}[1]{\textcolor[rgb]{0.25,0.44,0.63}{#1}}
\newcommand{\StringTok}[1]{\textcolor[rgb]{0.25,0.44,0.63}{#1}}
\newcommand{\VerbatimStringTok}[1]{\textcolor[rgb]{0.25,0.44,0.63}{#1}}
\newcommand{\SpecialStringTok}[1]{\textcolor[rgb]{0.73,0.40,0.53}{#1}}
\newcommand{\ImportTok}[1]{#1}
\newcommand{\CommentTok}[1]{\textcolor[rgb]{0.38,0.63,0.69}{\textit{#1}}}
\newcommand{\DocumentationTok}[1]{\textcolor[rgb]{0.73,0.13,0.13}{\textit{#1}}}
\newcommand{\AnnotationTok}[1]{\textcolor[rgb]{0.38,0.63,0.69}{\textbf{\textit{#1}}}}
\newcommand{\CommentVarTok}[1]{\textcolor[rgb]{0.38,0.63,0.69}{\textbf{\textit{#1}}}}
\newcommand{\OtherTok}[1]{\textcolor[rgb]{0.00,0.44,0.13}{#1}}
\newcommand{\FunctionTok}[1]{\textcolor[rgb]{0.02,0.16,0.49}{#1}}
\newcommand{\VariableTok}[1]{\textcolor[rgb]{0.10,0.09,0.49}{#1}}
\newcommand{\ControlFlowTok}[1]{\textcolor[rgb]{0.00,0.44,0.13}{\textbf{#1}}}
\newcommand{\OperatorTok}[1]{\textcolor[rgb]{0.40,0.40,0.40}{#1}}
\newcommand{\BuiltInTok}[1]{#1}
\newcommand{\ExtensionTok}[1]{#1}
\newcommand{\PreprocessorTok}[1]{\textcolor[rgb]{0.74,0.48,0.00}{#1}}
\newcommand{\AttributeTok}[1]{\textcolor[rgb]{0.49,0.56,0.16}{#1}}
\newcommand{\RegionMarkerTok}[1]{#1}
\newcommand{\InformationTok}[1]{\textcolor[rgb]{0.38,0.63,0.69}{\textbf{\textit{#1}}}}
\newcommand{\WarningTok}[1]{\textcolor[rgb]{0.38,0.63,0.69}{\textbf{\textit{#1}}}}
\newcommand{\AlertTok}[1]{\textcolor[rgb]{1.00,0.00,0.00}{\textbf{#1}}}
\newcommand{\ErrorTok}[1]{\textcolor[rgb]{1.00,0.00,0.00}{\textbf{#1}}}
\newcommand{\NormalTok}[1]{#1}
\usepackage{graphicx,grffile}
\makeatletter
\def\maxwidth{\ifdim\Gin@nat@width>\linewidth\linewidth\else\Gin@nat@width\fi}
\def\maxheight{\ifdim\Gin@nat@height>\textheight0.8\textheight\else\Gin@nat@height\fi}
\makeatother
% Scale images if necessary, so that they will not overflow the page
% margins by default, and it is still possible to overwrite the defaults
% using explicit options in \includegraphics[width, height, ...]{}
\setkeys{Gin}{width=\maxwidth,height=\maxheight,keepaspectratio}

% Prevent slide breaks in the middle of a paragraph:
\widowpenalties 1 10000
\raggedbottom

\AtBeginPart{
  \let\insertpartnumber\relax
  \let\partname\relax
  \frame{\partpage}
}
\AtBeginSection{
  \ifbibliography
  \else
    \let\insertsectionnumber\relax
    \let\sectionname\relax
    \frame{\sectionpage}
  \fi
}
\AtBeginSubsection{
  \let\insertsubsectionnumber\relax
  \let\subsectionname\relax
  \frame{\subsectionpage}
}

\setlength{\parindent}{0pt}
\setlength{\parskip}{6pt plus 2pt minus 1pt}
\setlength{\emergencystretch}{3em}  % prevent overfull lines
\providecommand{\tightlist}{%
  \setlength{\itemsep}{0pt}\setlength{\parskip}{0pt}}
\setcounter{secnumdepth}{0}
\newcommand{\columnsbegin}{\begin{columns}}
\newcommand{\columnsend}{\end{columns}}
\definecolor{headbg}{RGB}{61, 96, 103}
\definecolor{headfg}{RGB}{232, 239, 241}
\definecolor{lightgrey}{RGB}{230, 230, 230}
\setbeamercolor{headtitle}{fg=headfg,bg=headbg}
\setbeamercolor{headnav}{fg=headfg,bg=headbg}
\setbeamercolor{section in head/foot}{fg=headfg,bg=headbg}
\defbeamertemplate*{headline}{miniframes theme no subsection}{ \begin{beamercolorbox}[ht=2.5ex,dp=1.125ex, leftskip=.3cm,rightskip=.3cm plus1fil]{headtitle} {\usebeamerfont{title in head/foot}\insertshorttitle} \hfill \leavevmode{\usebeamerfont{author in head/foot}\insertshortauthor} \end{beamercolorbox} \begin{beamercolorbox}[colsep=1.5pt]{upper separation line head} \end{beamercolorbox} \begin{beamercolorbox}{headnav} \vskip2pt\textsc{\insertnavigation{\paperwidth}}\vskip2pt \end{beamercolorbox} \begin{beamercolorbox}[colsep=1.5pt]{lower separation line head} \end{beamercolorbox} }
\makeatletter\renewcommand{\@metropolis@frametitlestrut}{ \vphantom{ÁÉÍÓÚABCDEFGHIJKLMNOPQRSTUVWXYZabcdefghijklmnopqrstuvwxyz()1234567890} }\makeatother
\defbeamertemplate*{footline}{miniframes theme no subsection}{}
\beamertemplatenavigationsymbolsempty

\title{Reducción de la dimensionalidad en problemas de clasificación con Deep
Learning}
\subtitle{Análisis y propuesta de herramienta en R}
\author{Francisco David Charte Luque}
\institute{Universidad de Granada}
\date{Trabajo Fin de Grado}

\begin{document}
\frame{\titlepage}

\section{Fundamentos matemáticos}

\begin{frame}{SO(3)}

\begin{exampleblock}{Definición}
  Llamamos $O(n)$ al grupo de las \textbf{matrices ortogonales} reales
  de dimensiones $n \times n$ con la operación de composición; es
  decir,
  \[O(n) = \left\{ M \in GL(n) \mid MM^t=M^tM = I_n \right\}.\]
\end{exampleblock}

\begin{exampleblock}{Definición}
  Llamamos $SO(n)$ al \textbf{subgrupo de rotaciones}, definido como
  el subgrupo de $O(n)$ formado por aquellas matrices que tienen
  determinante $1$. Es decir,
  \[SO(n) = \left\{ M \in O(n) \mid \mathrm{det}(M)=1 \right\}.\]
\end{exampleblock}

\end{frame}

\begin{frame}{Propiedades topológicas de las rotaciones}

\begin{exampleblock}{Proposición}
  El grupo $SO(3)$, con la topología inducida desde la topología usual
  de las matrices, es homeomorfo al espacio proyectivo tridimensional
  $\mathbb{RP}^3$.
\end{exampleblock}

De este resultado se deduce que \(SO(3)\) es \textit{conexo} y que tiene
un grupo fundamental no trivial, no siendo, por tanto,
\textit{simplemente conexo}.

\end{frame}

\section{Ángulos de Euler}\label{uxe1ngulos-de-euler}

\begin{frame}{Definición}

\begin{exampleblock}{Definición}
  Se conoce como sistema de los \textbf{ángulos de Euler} al que
  utiliza tres ángulos de rotación sobre los ejes cartesianos para
  describir la orientación de un objeto respecto a un sistema de
  coordenadas fijo en el espacio.
\end{exampleblock}

\end{frame}

\begin{frame}{Descomposición}

\begin{equation*}
  \mathbf{R}_{\psi,z}
  \mathbf{R}_{\phi,y}
  \mathbf{R}_{\theta,x} =
\end{equation*}\begin{equation*}
  \begin{pmatrix}
    \cos \psi & -\sin \psi & 0 \\
    \sin \psi & \cos \psi & 0 \\
    0 & 0 & 1
  \end{pmatrix}\begin{pmatrix}
      \cos \phi & 0 & \sin \phi \\
      0 & 1 & 0 \\
      -\sin \phi & 0 & \cos \phi
    \end{pmatrix}\begin{pmatrix}
    1 & 0 & 0 \\
    0 & \cos \theta & -\sin \theta \\
    0 & \sin \theta & \cos \theta
  \end{pmatrix}
\end{equation*}

\end{frame}

\section{Cuaternios}\label{cuaternios}

\begin{frame}{Versores}

\columnsbegin
\column{0.5\textwidth}

Esta columna estará a la \textbf{izquierda}.

\column{0.5\textwidth}

Esta columna estará a la \textbf{derecha}.

\columnsend

\end{frame}

\begin{frame}{Cuaternios como esfera 4-dimensional}

Expresando

\begin{align*}
  1 \mapsto \begin{pmatrix} 1 & 0 \\ 0 & 1 \end{pmatrix},&\quad
  i \mapsto \begin{pmatrix} i & 0 \\ 0 &-i \end{pmatrix},\\
  j \mapsto \begin{pmatrix} 0 & 1 \\-1 & 0 \end{pmatrix},&\quad
  k \mapsto \begin{pmatrix} 0 & i \\ i & 0 \end{pmatrix},
\end{align*}

podemos ver cualquier elemento de \(SU(2)\) como un cuaternio unitario:
\(a + bi + cj + dk\) para \(a,b,c,d \in \mathbb{R}\) cumpliendo
\(a^2+b^2+c^2+d^2 = 1\), y viceversa. Así, \(SU(2)\cong S^3\)

\end{frame}

\begin{frame}{Recubrimiento del grupo de rotaciones}

\(S^3\) es un recubridor de dos hojas de \(\mathbb{RP}^3\) (de hecho, es
su recubridor universal) \(\Rightarrow\) Los cuaternios unitarios
recubren el espacio de rotaciones \(SO(3)\).

\end{frame}

\section{Realización práctica}\label{realizaciuxf3n-pruxe1ctica}

\begin{frame}{Uso de cuaternios}

\end{frame}

\begin{frame}[fragile]{Uso de cuaternios}

\tiny

\begin{Shaded}
\begin{Highlighting}[]
\KeywordTok{var}\NormalTok{ t }\OperatorTok{:}\NormalTok{ float}\OperatorTok{;}
\KeywordTok{var}\NormalTok{ capsule }\OperatorTok{:}\NormalTok{ GameObject}\OperatorTok{;}

\CommentTok{// ...}

\ControlFlowTok{if}\NormalTok{ (enable_anim) }\OperatorTok{\{}
  \CommentTok{// update animation}
\NormalTok{  t }\OperatorTok{+=} \FloatTok{0.025}\OperatorTok{;}
  \VariableTok{capsule}\NormalTok{.}\VariableTok{transform}\NormalTok{.}\AttributeTok{rotation} \OperatorTok{=} \VariableTok{Quaternion}\NormalTok{.}\AttributeTok{Slerp}\NormalTok{(initial}\OperatorTok{,}\NormalTok{ end}\OperatorTok{,}\NormalTok{ t)}\OperatorTok{;}

  \CommentTok{// stop animation when finished}
  \ControlFlowTok{if}\NormalTok{ (t }\OperatorTok{>=} \DecValTok{1}\NormalTok{) }\OperatorTok{\{}
\NormalTok{    enable_anim }\OperatorTok{=} \KeywordTok{false}\OperatorTok{;}
\NormalTok{    t }\OperatorTok{=} \DecValTok{0}\OperatorTok{;}
  \OperatorTok{\}}
\OperatorTok{\}}
\end{Highlighting}
\end{Shaded}

\normalsize

\end{frame}

\section{}\label{section}

\begin{frame}{Referencias}

\begin{thebibliography}{9}

\bibitem{gelfand63}
  Gelfand, I.M.; Minlos, R.A.; Shapiro, Z.Ya. (1963),
  Representations of the Rotation and Lorentz Groups and their Applications,
  New York: Pergamon Press.

\bibitem{vince11}
  Vince, John (2011),
  Quaternions for Computer Graphics,
  Springer-Verlag London.

\bibitem{aluffi07}
  Aluffi, Paolo (2007),
  Algebra: Chapter 0,
  Graduate Studies in Mathematics. American Mathematical Society

\bibitem{unity}
  Unity Game Engine,
  Sitio Oficial,
  https://unity3d.com

\end{thebibliography}

\end{frame}

\end{document}
